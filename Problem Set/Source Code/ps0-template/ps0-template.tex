%
% 6.006 problem set 0 solutions template
%
\documentclass[12pt,twoside]{article}

\input{macros-sp20}
\newcommand{\theproblemsetnum}{0}

\title{6.006 Problem Set 0}

\begin{document}

\handout{Problem Set \theproblemsetnum}

\setlength{\parindent}{0pt}
\medskip\hrulefill\medskip

{\bf Name:} Ziyou Ren

\medskip\hrulefill

%%%%%%%%%%%%%%%%%%%%%%%%%%%%%%%%%%%%%%%%%%%%%%%%%%%%%
% See below for common and useful latex constructs. %
%%%%%%%%%%%%%%%%%%%%%%%%%%%%%%%%%%%%%%%%%%%%%%%%%%%%%

% Some useful commands:
% $f(x) = \Theta(x)$
% $T(x, y) \leq \log(x) + 2^y + \binom{2n}{n}$
% \ttt{code\_function}


% You can create unnumbered lists as follows:
% \begin{itemize}
%     \item First item in a list
%         \begin{itemize}
%             \item First item in a list
%                 \begin{itemize}
%                     \item First item in a list
%                     \item Second item in a list
%                 \end{itemize}
%             \item Second item in a list
%         \end{itemize}
%     \item Second item in a list
% \end{itemize}

% You can create numbered lists as follows:
% \begin{enumerate}
%     \item First item in a list
%     \item Second item in a list
%     \item Third item in a list
% \end{enumerate}

% You can write aligned equations as follows:
% \begin{align}
%     \begin{split}
%         (x+y)^3 &= (x+y)^2(x+y) \\
%                 &= (x^2+2xy+y^2)(x+y) \\
%                 &= (x^3+2x^2y+xy^2) + (x^2y+2xy^2+y^3) \\
%                 &= x^3+3x^2y+3xy^2+y^3
%     \end{split}
% \end{align}

% You can create grids/matrices as follows:
% \begin{align}
%     A =
%     \begin{bmatrix}
%         A_{11} & A_{21} \\
%         A_{21} & A_{22}
%     \end{bmatrix}
% \end{align}

\begin{problems}

\problem % Problem 1

\begin{problemparts}
\problempart % Problem 1a
$A=\{1,6,12,13,9\}$\\
$B=\{3,6,12,15\}$\\
$A\cap B=\{6,12\}$
\problempart % Problem 1b
$|A\cup B|=|\{1,3,6,12,13,15,9\}|=7$
\problempart % Problem 1c
$|A-B|=|\{1,13,9\}|=3$
\end{problemparts}

\problem  % Problem 2

\begin{problemparts}
\problempart % Problem 2a
\begin{tabular}{|c|c|c|c|c|}
	\hline X&0&1&2&3\\
	\hline Probability&$\frac{1}{8}$&$\frac{3}{8}$&$\frac{3}{8}$&$\frac{1}{8}$\\
	\hline
\end{tabular}\\
So, the expectation of the random variable X should be 

$E[X] = 0\times\frac{1}{8}+1\times\frac{3}{8}+2\times\frac{3}{8}+3\times \frac{1}{8}$\\
$ = \frac{3}{2}$
\problempart % Problem 2b
The sample space of Y seems very complicated, but we got tricks here.
\begin{align}
    \begin{split}
            E[Y] &= \frac{1}{36}\times(1\times 21+\ldots +6\times21) \\
                &= \frac{1}{36}\times((1+\ldots+6)\times 21) \\
                &= \frac{1}{36}\times21\times21 \\
                &= \frac{49}{4} \\
                &= 12.25
    \end{split}
\end{align}
\problempart % Problem 2c
Since X is independent to Y, and the following formula holds.
$$E[X+Y]=E[X]+E[Y]$$
We got: $E[X+Y]=\frac{55}{4}=13.75$
\end{problemparts}

\newpage
\problem  % Problem 3

\begin{problemparts}
\problempart % Problem 3a
If $B=60mod42$, it means that B has a remainder of 60 after divided by 42. Thereby, we can write B as $B=42k+60,k=\ldots,-1,0,1,\ldots$\\
We can easily see that A = 100, which can be divided by 2 and B is also divisible by 2. Thus, $A\equiv B\equiv 0(mod\ 2)$
\problempart % Problem 3b
$B\equiv 0(mod\ 3)$, while $A\equiv 1(mod\ 3)$\\
Thereby, $A\not\equiv B(mod\ 3)$
\problempart % Problem 3c
$B\equiv 2(mod\ 4)$, while $A\equiv 0(mod\ 4)$\\
Thereby, $A\not\equiv B(mod\ 4)$
\end{problemparts}

\problem\\  % Problem 4
\textbf{Base Case}: $n=1$\\
We got $\sum\limits_{i=1}^{1}i^3=1$\\
$[\frac{n(n+1)}{2}]^2 = 1$\\
The base case holds.\\
\textbf{Inductive Step}: $n\longrightarrow n+1$\\
\begin{align}
    \begin{split}
            \sum\limits_{i=1}^{n+1}i^3 &= \sum\limits_{i=1}^{n}i^3+(n+1)^3 \\
                &= \frac{n(n+1)}{2}]^2+(n+1)^3 \\
                &= \frac{n^2(n+1)^2}{4}+(n+1)^3 \\
                &= (n+1)^2[\frac{n^2}{4}+n+1] \\
                &= [\frac{(n+2)(n+1)}{2}]^2 \\
                &= \sum\limits_{i=1}^{n+1}i^3
    \end{split}
\end{align}
\textbf{Q.E.D.}
\newpage
\problem\\  % Problem 5
Here, we inducted on the number of nodes i.e. $|V|$.\\
\textbf{Base Case}: $|V|=1$\\
$|E|=|V|-1=0$\\
Since there is no edge at all, a cycle can not form.\\
\textbf{Inductive Step}: $n\longrightarrow n+1$\\
We can easily prove this by contradiction.\\
That is we assume adding another vertex and edge into the graph G will engender a cycle while the condition still holds. Obviously, we can only achieve a cycle by adding this edge to any two vertices of this connected component.\\
However, this would bring our newly added vertex unconnected, which violates the condition that this graph is connected. Thereby, the inductive step holds.\\
\textbf{Q.E.D.}


\vfill
\problem  % Problem 6
Submit your implementation to {\small\url{alg.mit.edu}}.\\
Here we implement a more general algorithm. For implementation convenience, requirements as as following: \\
numpy \\
collections \\

\begin{lstlisting}
def count_long_subarray(A):
    '''
    Input:  A     | Python Tuple of positive integers
    Output: count | number of longest increasing subarrays of A
    '''
    count = 0
    ##################
    # YOUR CODE HERE #
    list_count = []
    temp = 0
    increasing_count = 0
    for idx, item in enumerate(A):
        if item > temp:
            increasing_count += 1

        else:
            list_count.append(increasing_count)
            increasing_count = 1

        temp = item

    list_count.append(increasing_count) # get all the length of increasing subarrays

    max_num = np.max(np.array(list_count)) # find the maximum of subarrays' lengths
    dict = collections.Counter(list_count) # count numbers
    count = dict[max_num]
    ##################
    return count
\end{lstlisting}

\end{problems}

\end{document}
